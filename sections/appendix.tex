\newpage
\begin{appendix}
\chapter{Anhang}
\section{Interview mit Samuel Müller}
\begin{itemize}
    \item \textbf{Wie sieht dein üblicher Prozess aus, wenn du mit Freunden spielen möchtest?}\\
    In die (WhatsApp) Gruppe schreiben und dann hoffen, dass sich rechtzeitig genug Leute melden. Danach muss man zuerst einmal Fragen wo, dann was und wenn man Glück hat antworten die Leute rechtzeitig. Das wäre der Einfache Prozess aber meistens läuft es sehr spontan ab.
    \item \textbf{Wie oft funktioniert dieser Prozess?}\\
    Wenn ich ehrlich bin, mittlerweile nicht mehr sehr oft. Früher sehr oft aber mittlerweile, wo jeder etwas anderes macht und vor allem gerade jetzt im Sommer - ich weiss nicht ob es am Sommer liegt - ist es relativ anstrengend etwas zu finden. Früher war es einfacher, auch im Winter und allgemein.
    \item \textbf{Gibt es da Probleme? Wenn ja welche Probleme?}\\
    Eine Sache ist sicher, dass man oft sagt, dass man spontan schaut und sich nicht sicher ist. Ein anderes Problem ist, dass es oft zu lange dauert, bis sich die Leute melden. Wenn sich die Leute erst in der letzten Sekunde melden, ist es blöd, weil man dann nicht planen kann. Auch blöd ist, dass man immer nachfragen muss. Zuerst muss man fragen, wer alles Lust hat, dann fragt man wo und dann was. Das wäre sicher etwas, was man einfacher machen könnte.
    \item \textbf{Was genau findest Du an der Organisationsweise über Whatsapp mühsam und weshalb?}\\
    \item \textbf{Du kennst sicher Doodle. Was hat Doodle nicht, dass Du es nicht für die Koordination von Spielen nutzen kannst?}\\
    Ich finde, das Doodle das perfekte Beispiel für Planungen ist und ich bin kein Fan von Doodle, weil man 10 Termine eingibt und dann warten muss bis alle anderen ihre Termine angegeben haben und in dieser Zeit müsste man eigentlich 10 Termine blockieren und das ist dann einfach sehr mühsam.
    \newline
    Giancarlo: Ja stimmt, und oftmals vergehen dann mehrere Wochen und man vergisst wieder, dass man die Termine angegeben hat und schlussendlich verplant man es dann doch.
    \newline
    Samuel: Stimmt und ein anderer Punkt ist, dass ich Doodle mittlerweile nicht mehr sehr handlich finde, vor allem wegen den Werbungs-Anzeigen mittendrin.

    \item \textbf{Wärst du auch bereit, an einer Session mit fremden Spielern teilzunehmen? Falls nicht, wieso?}\\
    Es ist beides in Ordnung. Wenn es geht, dann natürlich mit meinen Kollegen aber ich habe auch kein Problem mit irgendwelchen Leuten zusammen zu sitzen.
    \newline
    Giancarlo: Was noch gut wäre, wenn man angeben könnte, welches Spiel man spielen möchte und dann spielst Du mit Leuten zusammen, die das Spiel schon kennen, dann musst Du das Spiel nicht zuerst noch erklären. Dann hättest Du zudem noch eine Herausforderung, wenn Du gegen erfahrene Spiele spielst.
    \newline
    Samuel: Man könnte auch so etwas wie ein “Kenner”-Level machen. Und, dass jeder Benutzer angeben kann, was er gerne Spielt und sein Hauptspiel angeben kann. Aber es gibt auch Spiele, die cool sind mit neuen Leuten, z.B. das “Werwölfe” Spiel, das sehr interessant sein kann mit sehr vielen und fremden Leuten. Es kommt sehr auf das Spiel darauf an. Zum Beispiel bei “Mega Safe” ist es eine Katastrophe, wenn jemand die Regeln nicht kennt.
    \newline
    Giancarlo: Stimmt, vor allem bei den komplexen Spielen, bei denen man schon nur eineinhalb Stunden spielen muss, bis man ein einigermassen gutes Verständnis des Spiels erlangt.
    \newline
    Sämi: Und auch wegen der Spiellänge. Wenn ein Spiel 4 Stunden dauert und man zuerst noch eineinhalb Stunden das Spiel erklären muss, ist das einfach mühsam.
    \item \textbf{Ist dir beim Spielen der Aspekt des ``mit Freunden Zeit verbringen`` oder der ``Strategischen/Gedanklichen Challenge`` wichtiger?}\\
    Eine Mischung aus beiden und es ist auch vom Spiel abhängig. Beim “Werwölfe” Spiel kommt es mir nicht so auf das Strategische darauf an, da möchte ich einfach mit meinen Freunden Zeit verbringen. Aber wenn ich etwas komplexes, wie “Civilisation” spiele, dann hat das Strategische schon eher vorrang.
    \item \textbf{App oder Mobile Website? Mit was arbeitest du lieber?}\\
    Definitiv eine App. Ersten, weil man so schnell wie möglich zusagen möchte und da ist eine App einfacher, vor allem auch wenn man unterwegs ist. Zwar sind Webseiten auf dem Handy oft übersichtlich, aber ich bevorzuge trotzdem Apps.
    \newline
    Giancarlo: Klar, und mit einer App könnte man auch einfach Notifications verwenden, wo man dann direkt reagieren kann.
    \newline
    Samuel: Mittlerweile können sogar Webseiten Notifications senden.
    \item \textbf{Würdest Du es gut finden, wenn man angeben könnte, dass man lieber einfache oder komplexe Spiele spielen möchte?}\\
    Ja, das wäre sicher eine Option und die wäre sicherlich auch lustig. Es hat dann auch einen Einfluss auf den Spielstil, zum Beispiel wenn man es eher locker nimmt, dann lässt man sich viel Zeit bei den Runden aber andererseits, wenn man es eher strategisch mag, möchte man schon vorwärts kommen und weniger Zeit verschwenden.
    \item \textbf{Was frustriert dich am meisten, wenn es um die Organisation einer Brettspiel Session geht? Hast du ein konkretes Beispiel?}\\
    Ja, zum Beispiel wenn 2 von 4 Leuten sagen, sie schauen spontan ob sie dabei sind oder nicht. Oder wenn jemand zusagt und später schreibt, man könnte auch ein anderes Spiel spielen und dann finden sie vielleicht andere Personen in der Gruppe, die auch das andere Spiel lieber spielen und schlussendlich sagen sie das erste Spiel ab..
    \newline
    Giancarlo: Ein anderes Problem finde ich auch, dass es oft vorkommt, dass die WhatsApp Gruppe oft für andere Aktivitäten “missbraucht” wird und es nicht nur um Brettspiele geht. Zum Beispiel wird auch gefragt, ob jemand Lust hat, morgen Volleyball zu spielen oder an einen See fahren möchte. Denkst Du, dass man dies ein wenig trennen könnte, wenn man eine App hat, die spezifisch auf Brettspiele fokussiert ist, da man dort zusagen oder absagen muss und man nicht “vielleicht” angeben kann?
    \newline
    Samuel: Ja das “vielleicht” sollte sicherlich nicht drin sein und man sollte auch fast so etwas wie eine Sperre einrichten für Spieler die Zusagen und dann doch nicht erscheinen.
    \item \textbf{Fällt dir ein Mechanismus ein, welcher die Spieler motivieren würde, um mehr an Sessions teilzunehmen?}\\
    Da fallen mir spontan Levels ein, die man erreichen kann indem man an vielen Sessions teilgenommen hat. Oder eine Art Medaillen, die man erhält, wenn man z.B. schon an 100 Sessions teilgenommen hat.
    \newline
    Giancarlo: Stimmt, und das passt sicherlich noch zu den Personen, die Brettspiele spielen, da die ein höheres Verständnis für solche “Game-Features” haben.
    \newline
    Samuel: Man könnte auch Punkte verteilen, wenn jemand zum Beispiel schon 10 Mal gewonnen hat in 12 Teilnahmen. Natürlich kann man das nicht übergreifend auf alle Benutzer machen, da man ansonsten einfach Betrügen kann. Aber für sich selber wäre es toll, Statistiken zu haben.
    \newline
    Giancarlo: Ja, oder Ranglisten. Es wäre auch gut, wenn man sehen könnte wie gut ein fremder Spieler ist und Du dir dann überlegen könntest, ob Du mit dieser Person spielen möchtest.
    \newline
    Samuel: Ja stimmt und das könnte man dann weiterführen, indem man Freundeslisten haben kann, wo man dann für ein privates Spiel nur die besten Spieler einladen kann.
    \item \textbf{Würdest Du es noch gut finden, wenn man ein Spiel als privat oder als öffentlich deklarieren könnte?}\\
    Gute Idee; Man sollte aber auch Personen/Freunde auswählen können und nicht nur alle seine Freunde.
    \item \textbf{Was wäre für dich \underline{die} Hauptfunktion der Software?}\\
    Das Organisatorische mit fixen Zu- und Absagen. Man gibt einen Raum, eine Zeit ein Spiel und die Spieler an und dann können die Spieler zu- oder absagen.
    \item \textbf{Was ist das Erste was du machst, wenn du Lust hast, ein Brettspiel mit Freunden zu spielen?}\\
    \item \textbf{Wie entscheidest du / deine Gruppe, welches Spiel gespielt werden soll? Hast du eine Idee, wie man diesen Prozess besser gestalten könnte?}\\
    Jemand macht Vorschläge in der Gruppe und man diskutiert und entscheidet gemeinsam.
    \item \textbf{Lösung für Problem, dass wenn jemand nicht kommen kann, der ein Spiel besitzt?}\\
    1. Man kann für eine Session mit 4 Personen einladen und angeben, welche Spiele infrage kämen und dann könnten alle eingeladenen Personen angeben, welches Spiel sie spielen möchten und das Spiel, das zuerst 4 Personen hat, wird dann gespielt.
    \newline
    2. Jeder User kann im Profil hinterlegen, welche Spiele er besitzt und dann kann man für eine Session einladen und wenn dann ein Spieler für die Session zusagt, werden seine Spiele als Vorschläge für die Session aufgelistet und es kann abgestimmt werden, welches Spiel gespielt werden soll. Problem: Oftmals sagen Spieler nur für eine Session, unter der Bedingung, dass ein bestimmtes Spiel gespielt wird, zu.
    \newline
    3. Man könnte verschieden Arten von Sessions machen. Z.B. “Pro-Spiel”, ein spezifisches Spiel oder “Open for all”. Problem: Wenn jemand ein “Pro-Spiel” spielen möchte und die anderen möchten alle ein Spiel spielen, das er nicht gerne spielt. Lösung: Jeder kann im Profil sogenannte “Veto-Spiele” hinterlegen, die er auf keinen Fall spielen möchte und diese würden dann in der Session gar nicht aufgelistet werden, wenn er eine Session betritt. Oder man gibt eine Auswahl von 3 Spielen an, die man gerne spielt und wenn dann nicht sein Lieblingsspiel gespielt wird, muss man trotzdem bei den anderen mitmachen.
    \item \textbf{Was muss für Dich die App sicher haben, damit Du sie a) verwendest und b) weiterempfiehlst?}\\
    Kalender, bei dem es ersichtlich ist, wo man noch frei ist und wo nicht. Muss verknüpfbar mit privatem Kalender sein;
    \item \textbf{Darf eine App etwas kosten für Dich? Oder hast Du lieber Werbung in der App?}\\
    Egal, aber es wird entsprechender Support erwartet. Bei bezahlten Apps mehr als bei gratis Apps mit Werbung.
    \item \textbf{Zähle 5 Adjektive auf, mit denen Du Apps beschreiben würdest, die du toll findest. (evtl. mit Begründung und Nennung des Erlebnisses)}\\
    übersichtlich, einfach handzuhaben, schnell (performant), datenverbrauch (keine hintergrunddaten), sozial
    \item \textbf{Zähle 5 Funktionen auf, die Du unbedingt in der App haben willst. (must have)}\\
    Kalenderanbindung, übersichtlich; Menü/Startseite;
    \item \textbf{Ist es Dir wichtig, dass Du in einer App eine bestimmte Farbe oder ein Hintergrundbild oder Profilbild hinterlegen kannst? (Personalisierung)}\\
    Nebensächlich, Eventuell ein Profilbild, aber die Übersichtlichkeit leidet schnell. Nutzen ist wichtiger.
    \item \textbf{Wärst Du bereit für eine bessere Organisationweise auf eine andere App/Webseite zu wechseln?}\\
    Ja, und die Whatsappgruppe anderweitig nutzen. Weniger inaktive Leute.
    \item \textbf{Was sind die drei Hauptfeatures, die die neue App/Webseite mitbringen soll, damit Du Dir einen Wechsel überlegst?}\\
    \item \textbf{Was sind die Gründe weshalb in der Vergangenheit Sessions nicht zustande gekommen sind? Was hat gefehlt?}\\
    \item \textbf{Ist Dir wichtiger, dass überhaupt eine Session zustande kommt oder dass die richtigen Leute in der Session sind? (Quality vs. Quantity) Weshalb?}\\
    Leute die Brettspiele spielen sind oft sozialer und dann spielt es weniger eine Rolle, wer dabei ist. Aber es gibt Ausnahmen aber die lösen sich von alleine, indem eine Person einfach nicht die Zusage zu einer Session gibt, bei der eine andere Person dabei ist, die sie nicht mag.
    \item \textbf{Sonstiges?}\\
    Vermarktung? Kommt darauf an wie gut es wird. Interesse sicher vorhanden.
    \newline
    Möglichkeit neue soziale Kontakte zu knüpfen.
    \newline
    Kalenderanbindung kann schwer sein.
    \newline
    Man kann auch Sessions mit alternativen Zeiten vorschlagen und dann die Session mit mehr Teilnehmern durchführen. Wichtig dabei ist auch einen Zeitpunkt für die Deadline der Anmeldungen zu setzen, damit man besser planen kann.
    \newline
    Deadline, wann?
    \newline
    72 Stunden vorher. 24 Stunden zu knapp.
    \newline
    Ort wird oft nicht von Initiator vorgeschlagen, wie lösen?
    \newline
    Einen Ort vorschlagen; Initiator und Teilnehmer müssen angeben, ob sie eine Location zur Verfügung stellen können oder nicht. Wenn niemand eine Location hat, wird die Session logischerweise nicht durchgeführt. App prüft nach ablauf der Deadline alle Parameter und gibt per Notification bescheid, ob die Session zustande kommt oder nicht. App könnte auch fragen, ob man wirklich keine Location hat und man dann entweder eine neue Location angeben oder die Session definitiv absagen kann. Wenn mehrere eine Location anbieten können, könnte man entweder per Zufall die Location auswählen oder die Location wählen, die am zentralsten liegt.
    \newline
    Chatfunktion wäre sicherlich nicht schlecht, aber macht die App vielleicht wieder zu überladen. Fraglich ist, ob ein Chat überhaupt nötig ist, da man ja drei von vier Fragen, die für die Session nötig sind, schon beantwortet hat mit dem Erstellen der Session; nämlich das Was, Wo und Wann.
    \newline
    Was passiert, wenn der Spielleiter kurzfristig ausfällt (z.B. krankheitshalber)?
    \newline
    Roter Button: “Austreten”, mit Pop-Up: “Bist Du sicher, dass Du austreten möchtest?”, mit Textbox um den Grund anzugeben. Danach können die restlichen Teilnehmer dem austretenden Mitglied Punkte abziehen, wenn sie möchten (durch eine Abstimmung). Vielleicht auch mit Achievement: “Quitter” bei wiederholtem auftreten. Restliche Teilnehmer könnten dann auch entscheiden, ob sie eine neue Session starten möchten und eine Ersatzperson einladen möchten. Wenn jemand schon das Label “Quitter” besitzt und eine weitere Session absagt, sollte er eine Strafe erhalten, z.B. einen Monat lang, darf er bei keiner Session zusagen.
    \newline
    Darf der Host noch darüber entscheiden, wer in der Session dabei sein darf oder ist sie offen für alle?
    \newline
    Weniger nötig, solange es die Möglichkeit gibt private und öffentliche Sessions zu machen.
    \newline
    Es braucht sicher noch die Möglichkeit um Probleme mit der App den Entwicklern zu melden.
\end{itemize}
\end{appendix}