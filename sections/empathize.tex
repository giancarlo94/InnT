\section{Empathize}
\subsection{Beschreibung der Interviews}
\subsubsection{Erarbeiten der Interviewfragen}
Die Interviewfragen wurden von allen unabhängig gesucht, wobei jeder sechs Fragen einreichen musste. Bei den Fragen wurde speziell darauf geachtet, möglichst viel und detailiert über das Organisieren von Spielsessions herauszufinden. Besonderer Fokus wurde auch auf Probleme gelegt, die die momentane Organisationsform beeinträchtigen.
\subsubsection{Verhalten während des Interviews}
Während des Interviews wurde dem Interviewten, nebst den Fragen, auch die Möglichkeit gegeben, eigene Lösungsvorschläge für bestehende Probleme einzubringen, welche dann gemeinsam ausdiskutiert wurden. Dies ist ein wesentlicher Teil der Empathize Phase, denn dadurch lernt der Entwickler die Denkweise des potenziellen Kunden kennen und erhält besonders tiefe Einblicke in den Fachbereich.


\subsection{Stakeholderanalyse}
In diesem Abschnitt werden die verschiedenen Stakeholder und deren Eigenschaften
aufgeführt. Die als Kontakt angegebenen Personen sind der Vollständigkeit halber 
angegeben. Alle Gruppenmitglieder haben die Rollen der Entwickler und der 
Softwarearchitekten.

\vspace{20pt}
\begin{tabular}{p{3cm}|p{9cm}}
     {\large\textbf{Funktion}} & {\large\textbf{Entwickler}}\\
     \hline
     \textbf{Verfügbarkeit} & Gut \\
     \hline
     \textbf{Einfluss} & Hoch \\
     \hline
     \textbf{Haltung}&Sehr positiv\\
     \hline
     \textbf{Wissensgebiet} & Android Entwicklung, Material Design, Datenbankwissen, UML\\
     \hline
     \textbf{Ziele} & Eine funktionsfähige Applikation mit modernen Methoden und Technologien zu erstellen\\
     \hline
     \textbf{Kontakt}&\textbf{Developer:}\\
    &Yvo Brönnimann:\\
   	&yvo.broennimann@gmail.com\\
    &Zeitliche Verfügbarkeit:\\
    &Dienstag bis Donnerstag von 9:00 bis 17:00 Uhr\\
    \hline
    \textbf{Relevanz für die Abnahme} & Hoch, die Entwickler sind gleichzeitig die Projektleiter, ihre Zufriedenheit mit dem System wirkt sich direkt auf die Abnahme aus.\\
    \hline

     
\end{tabular}
\vspace{30pt}\\


\begin{tabular}{p{3cm}|p{9cm}}
     {\large\textbf{Funktion}} & {\large\textbf{Softwarearchitekt}} \\
     \hline
     \textbf{Verfügbarkeit} & Gut \\
     \hline
     \textbf{Einfluss} & Hoch \\
     \hline
     \textbf{Haltung}&Sehr positiv\\
     \hline
     \textbf{Wissensgebiet} & Architekturpatterns, Designpatterns, Datenbankwissen,
     UML\\
     \hline
     \textbf{Ziele} & Eine sinngemässe Architektur für die Applikation zu erstellen\\
     \hline
     \textbf{Kontakt} & \textbf{Lead Architect:}\\
                &Giancarlo Bergamin:  \\
                &giancarlo.bergamin@swisscom.com\\
                &Zeitliche Verfügbarkeit: \\
                &Dienstag bis Donnerstag von 8:30 bis 17:00 Uhr\\
    \hline
    \textbf{Relevanz für die Abnahme} & Hoch, die Entwickler sind gleichzeitig die
    Projektleiter, ihre Zufriedenheit mit dem System wirkt sich direkt auf die Abnahme
    aus.\\
    \hline
    \end{tabular}
    \vspace{30pt}\\
    
\begin{tabular}{p{3cm}|p{9cm}}
     {\large\textbf{Funktion}} & {\large\textbf{Benutzer und gleichzeitig Auftraggeber}} \\
     \hline
     \textbf{Verfügbarkeit} & Mittel \\
     \hline
     \textbf{Einfluss} & Hoch \\
     \hline
     \textbf{Haltung}&Positiv\\
     \hline
     \textbf{Wissensgebiet} & Android Applikation bedienen, Brettspiel Domänenwissen\\
     \hline
     \textbf{Ziele} & Eine einfach zu bedienende Applikation erhalten, welche ihre Probleme löst und welche verlässlich funktioniert.\\
     \hline
     \textbf{Kontakt} & \textbf{Vertretung der Benutzer:}\\
                &Samuel Müller  \\
                &Zeitliche Verfügbarkeit: \\
                &Abends per WhatsApp erreichbar.\\
    \hline
    \textbf{Relevanz für die Abnahme} & Hoch, der Benutzer ist sehr an einer funktionierenden Lösung interessiert. Samuel hat das Problem identifiziert und übernimmt somit die Rolle des Auftraggebers.\\

     
\end{tabular}

\subsection{Zusammenfassung der Erfahrungen in dieser Phase}
Hier sollen unsere Erfahrungen in dieser ersten Phase zusammengefasst werden.
{\color{red}todo}

\iffalse
\subsection{Anforderungsanalyse}
Dadurch, dass Samuel Teil einer festen Brettspielgruppe ist, könnte man die Mitglieder der Gruppe befragen, welche Anforderungen sie für essentiell halten und welche Anforderungen nice-to-have wären. Mit gezielten Interviews der Gruppenmitglieder können so funktionale Anforderungen erhoben werden. Die nichtfunktionalen Anforderungen können von den Entwicklern festgelegt werden, da diese über das nötige technische Knowhow verfügen.
\fi