\section{Ideensuche}
Wir hatten initial mehrere Projektideen, welche sich aber nach kurzer Recherche als schon implementiert erwiesen haben.
In einem weiteren Schritt hat jeder unserer Gruppenmitglieder in seinem Freundeskreis nach Problemen gesucht, welche durch eine Softwarelösung gelöst werden könnten. 
Dabei haben wir ein Problem identifiziert, welches sich gut durch eine Mobile Applikation lösen lässt. Zunächst beschreiben wir in diesem Abschnitt vier Ideen, welche wir initial hatten, uns jedoch gegen diese entschieden haben. Der letzte Abschnitt beschreibt dann das Problem, auf welches wir uns schlussendlich geeinigt haben.

\subsection{Plattform für (Neo-)Veganer}
 Eine erste Idee war es zum Beispiel, eine Plattform für Veganer zu entwickeln, welche es erlaubt, vegane Restaurants und Shops zu identifizieren, um so den Einstieg in den Veganismus zu erleichtern. Die Systeme, welche wir jedoch gefunden haben, waren sehr durchdacht und gut zu bedienen. Weil wir kein grosses Verbesserungspotential bei diesen Plattformen gesehen haben, haben wir uns entschieden weiter zu suchen.\footnote{https://www.happycow.net/}
 
 \subsection{Plattform zur Suche von Hofladen}
 Eine weitere Idee entstand aus einem Problem, dass Yvo im Alltag begegnet ist: Er kauft gerne lokal ein und findet deshalb Hofladen sehr gut. Das Problem dabei ist es aber, dass Hofladen meistens keine Webseiten haben und auch sonst keine Werbung schalten, was es schwierig macht diese überhaupt aufzufinden. Das Problem könnte man mit einer Plattform zur Suche von Hofladen lösen. Auch diese Idee war leider schon implementiert und deshalb haben wir uns entschieden weiter zu suchen.\footnote{https://www.myfarm.ch/de/hoflaeden}
 
 \subsection{Finden von Praktika und Traineeprogrammen nach dem Studium}
 Als nächstes hatte Sandrine die Idee, eine Plattform zum Finden von Praktika oder Traineeprogrammen zu erstellen, um den Studenten den Einstieg in das Arbeitsleben zu erleichtern. Dieses Problem wurde auch schon von talendo.ch gelöst.\footnote{https://talendo.ch/de} Auch diese Lösung war schon sehr schön und gut implemetiert, deshalb haben wir weiter gesucht.
 
 \subsection{Skillfinder}
 Ein weiteres Problem, welchem Giancarlo schon begegnet ist, war die Suche nach geeigneten Leuten, um eine Idee umzusetzen. Wenn zum Beispiel Giancarlo als Software Entwickler eine Produktidee hätte, braucht er natürlich noch andere Personen mit anderen Skills wie zum Beispiel einen Designer, einen Betriebswirtschaftler oder eine Buchhalterin um die Idee umzusetzen und vielleicht sogar ein Startup zu gründen. Um diese Personen zu finden wäre eine Webplattform ideal. Auch dieses Problem wurde schon von mehreren Plattformen gelöst, zum Beispiel von cofounderslab.com.\footnote{https://cofounderslab.com/}
 
\subsection{Kartenspiel-Koordinationsapp}
Zu diesem Zeitpunkt gingen uns langsam die Ideen aus. Deshalb haben wir uns entschieden in unserem Freundeskreis nach Problemen und Bedürfnissen zu suchen. Dabei sind wir auf ein Problem gestossen, welches noch von niemandem zufriedenstellend gelöst wurde:
Samuel spielt sehr gerne Brett- und Kartenspiele mit seinen Freunden. Es ist jedoch sehr mühsam für jede Session einen Termin, einen Ort und das zu spielende Brettspiel zu organisieren und zu planen. Zudem wäre es cool, wenn einmal niemand seiner festen Brettspielgruppe Zeit hat, um sich zu treffen, bei einer anderen Gruppe mitzuspielen, welche vielleicht noch Spieler sucht, um eine Session starten zu können. 	
Mit der Applikation könnten Termine und Gastgeber gefunden werden und die Spieler könnten sich für potentielle Termine an- bzw. abmelden. Man könnte angeben, ob es eine private Session wird, sprich nur bestimmte Benutzer der App, welche zu einer festen Spielgruppe gehören teilnehmen dürfen oder ob sich jeder Benutzer der App für eine Session anmelden darf.  Zudem könnte die Applikation aufgrund der Spieleranzahl angeben, welche Spiele gespielt werden können und die Benutzer dann abstimmen lassen, welches Brettspiel gespielt werden soll. Eine weitere Möglichkeit wäre es eine Rangliste der besten Spieler zu erstellen, um zu sehen, wie fortgeschritten ein Spieler ist.
Eine Mobile App würde den bisherigen Prozess, welcher über eine WhatsApp Gruppe versucht Spieler und Termine zu finden, enorm vereinfachen.

Um das Problem und das sich darunter versteckte Bedürfnis besser zu verstehen, haben wir ein Interview mit Samuel Müller durchgeführt. Eine Transkription des Interviews ist im Anhang abgedruckt.